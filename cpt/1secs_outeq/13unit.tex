\section{Unitary time-evolution}



\subsection{Quench}

In the quench protocol, we define a family of Hamiltonians of type:
\ba{Hquench}
H(\bar \mu) = H_o + \bar\mu P \cm
\ea
where in this case the scaling variable $\bar \mu$ tunes the strength of the perturbation
$P$ such that $\Bigr[ H_o , P] \neq 0$ and $H_o$ is the unperturbed Hamiltonian 
whose parameters assume their critical values.

In this quench protocol, at $t = t_0$ the system starts in the ground state
of the Hamiltonian associated with an initial value $\bar \mu_i$ . Then, at time $t>0$, we
suddenly change the coupling from $\bar \mu$ to $\bar \mu _i$ and we follow the 
corresponding evolution of the system, described by the Schro\"odinger equation
in the form of the Von-Neumann equation:
\ba{eqschrodinger}
\partial_t \rho(t) = -i \Bigr[ \hat H(\bar \mu), \rho(t) \Bigr] \qquad \rho(t=0) = \rho_0 \,\,,
\ea
where $\rho(t)$ is the system density matrix at time $t\,$.

To express a possible scaling law, we define a further scaling variable associated
with the time:
\be{theta_din}
\theta = t \Delta \cm
\ee
which is obtained by recalling that the inverse energy difference of the lowest states
is proportional to the relevant time scale of the critical modes.

\subsection{Kibble-Zurek mechanism}

The Kibble-Zurek(KZ) mechanism is related to the amount of final defects after slow
passages through continuous transition, from disordered to the ordered phase 
\cite{kibble1976topology, kibble1980some, zurek1985cosmological, zurek1996cosmological, 
zurek2005dynamics}. This type of out-of-equilibrium process is several studied both
analytically-numericallly \cite{dziarmaga2010dynamics, PSSV-2011-noneqcoll,
chandran2012kibble, rossini2021coherent} both experimentally \cite{weiler2008spontaneous,
ulm2013observation}.

The large-scale modes, associated with the changes of the transition tuning parameter, are
insufficient to equilibrate the long-distance critical correlations. Even in the large
time variation regime, the out-of-equilibrium dynamics grows in the thermodynamic limit.
In other words, when the system evolves, starting from an equilibrium state, the
time-evolution is different from an adiabatic dynamics and the system does not pass 
through equilibrium states.

In this scenario, the out-of-equilibrium regime is always describable in terms of the RG
framework and the equilibrium scaling behaviors can be related by the quantum to classical
mapping \cite{rossini2021coherent, S99}.


