\section{Equilibrium Quantum Transitions}

\subsection{Continuous Quantum Transition (CQT)}
\label{CQTeq}

In the thermodynamic limit, i.e. infinite volume limit, same specific models could undergo a continuous phase transition. With the word phase, we intend the physical properties of the ground stare associated with the Hamiltonian and characterized by the values of the Hamiltonian parameters. If this phase changes driving one of this parameter, we are in the presence of a quantum transition and the point, in which that happens, is called transition point.

In the particular case of the CQT, the quantum ground-state properties are continuous close to the transition point, called also critical point. 
In this special point, the system develops a long-distance correlations and its microscopic behavior becomes negligible.

Independent from the local details, the global properties determine a notable universal critical behavior in which we collect different physical systems in universality class.
In terms of the Renormalization Group (RG) theory, the critical point and its universal behavior ase associated the fixed points of the RG flux. Hence, if we call $b$ the characteristic unit length of the system and if we define the RG transformation as the parameter rescaling respect $b$, the physical observables satisfy general scaling law unchanged along all the RG flux \cite{S99, C-1996-ScalingandRG}.

From a physics point of view, we can interpret the factor $b$ like the spatial correlation length $\xi$ which diverges approching the critical point.
According to the renormalization group (RG)
theory of critical phenomena, these global properties may be the spatial dimension-
ality, the nature of the order parameter, the symmetry and the symmetry-breaking
pattern~\cite{PV2002}.

Moreover, at CQTs the systems develop an equilibrium and dynamic scaling be-
havior in the thermodynamic limit and their quantum functions satisfy scaling
power laws characterized by universal critical exponents~\cite{CV2014}.

\subsection{First Order Quantum Transition (FOQT)}
\label{FOQTeq}

Quasi-degenerate vacua naturally arise in the context of quantum phase transitions, after a spontaneous symmetry breaking. Their behavior and coexistence in the non-critical regime is governed by a first-order quantum transition.

It is characterized by the crossings of the lowest-energy states in the infinite-volume limit and in the absence of conservation laws \cite{plissetto2023scaling}.
Instead, in a finite system, the energy gap among these states remain different from zero, giving rise to the phenomenon of avoided level crossing.

FOQTs are associated with many important out-of-equilibrium effects s, including nucleations and metastability \cite{binder1987theory,bray2002theory}, coarsening \cite{chandran2012kibble}, and anomalous dependence on the boundary conditions \cite{pelissetto2020scaling,panagopoulos2018dynamic,campostrini2015quantum,pelissetto2018finite,rossini2018ground}.

\subsection{Models}
\label{Models}

\subsubsection{Quantum Ising}
\label{QIsing}

As a first toy model for the study of quantum phase transitions, we consider the quantum 1D Ising Model whose Hamiltonian is given by:

\be{HIsing}
	H(g, h) = - \sum_{x=1}^{L-1} \sigma^{(1)}_x \sigma^{(1)}_{x+1}
	- h \sum_{x=1}^L \sigma^{(1)}_x - g \sum_{x=1}^L \sigma^{(3)}_x \pc
\ee

where $L$ is the system size and $\sigma_x^{(k)}$ are the Pauli matrices on the 
$x^{\rm th}$ site.\\

This system develops a quantum critical behavior at $g = g_c = 1$
and $h=0$, belonging to the 2D Classical Ising universality class \cite{S99}. Instead, 
when $h$ is different from zero, the lowest states energy gap is not vanish.

Along the RG flux, the relevant parameters associated with the RG perturbations at the 
fixed point are $r = g-g_c$ and $h$. Their RG dimensions are respectively $y_r=1/\nu=1$
and $y_h = 15/8$, so that the length scale $\xi$ of the critical modes behaves as
$\xi \sim \abs{g-g_c}^{-1/y_r}$ for $h=0$, and $\xi \sim \abs{h}^{-1/y_h}$ for $g=g_c$.
The dynamic exponent $z$ associated with the vanish critical gap $\Delta \sim \xi^{-z}$
at the transition point, is given by $z=1$. The order parameter field, which distinguishes 
the two phases and is associated with the longitudinal operators $\sigma_x^{(1)}$, has a
RG dimension equal to $y_l = d+z-y_h=1/8$, while that associated with the transverse
operator $\sigma_x^{(3)}$ is equal to $y_t = d+z-y_r = 1$.\\

In the ferromagnetic phase $g<1$, the model undergoes a FOQT at $h=0$. Across this point,
the system remains non-critical and displays exponential decay of the correlation 
functions.

For $h=0$, the model presents a level-crossing of the two lowest-energy state in the
infinite volume limit, where the energy gap closes exponentially for $L\to \infty$, e.g.
for open boundary condition \cite{cabrera1987role}:
\be{isingfoqtgap}
	\Delta(g,L) = 2 g^L(1-g^2) \Bigr[ 1+ {\cal O}(g^{2L}) \Bigr] \pt
\ee
In the limit of small longitudinal magnetic field, i.e. $\abs{h \ll 1}$, the system
presents a Zeeman-like gap in energy between the 2 lowest states which introduces another
symmetry-breaking of the degeneracy. We can express an approximation of this gap using
standard perturbation theory in $h$ \cite{campostrini2014finite}:
\be{isingperthgap}
	{\cal E}(g,h\to 0,L) \simeq 2 h \sum_{j=1}^{L}
	\abs{\braket{\sigma_j^{(3)}}} \simeq 2hLM_0(g) \cm
\ee
with $M_0 = (1-g^2)^{1/8}$ the approximated longitudinal magnetization in the constrains
$h = 0$ and $L\to \infty$.



\subsubsection{Kitaev chain}
\label{kitaev}

Now, we introduce the following lattice model, called Kitaev model \cite{Kitaev_2001},
whose Hamiltonian describes the fermions interaction with the lattice and is given by:
\ba{Hkitaev}
    \hat{H}&=-\sum_{x=1}^{L-1} (\hat{c}^{\dagger}_{x}\hat{c}_{x+1} 
    + \hat{c}^\dagger_{x}\hat{c}^\dagger_{x+1}+{\rm h.c}.)
    - \mu\sum_{x=1}^{L} \hat{n}_x\,,
    %\label{Hkitaev}
\ea
where $\hat{n}_x\equiv\hat{c}^\dagger_x\hat{c}_x$ is the number operator on the site $x$, and the operators $\hat{c}_x, \hat{c}^\dagger_x$ satisfy the canonical anticommutation relations, thus $\{\hat{c}_x, \hat{c}_y\}=\{\hat{c}^\dagger_x, \hat{c}^\dagger_y\}=0$ and $\{\hat{c}_x, \hat{c}^\dagger_y\}=\delta_{xy}$. Applying the Jordan-Wigner transformation~\cite{S99}, the Kitaev ring can be exactly mapped into a quantum Ising chain with a transverse field~\cite{P-1970-Isingmodel}. We point out that the transformation does not preserve also the same boundary conditions, so attention should be paid when recasting Eq.~\eqref{Hkitaev} in its bosonic counterpart~\cite{RV-2021-coherentanddissipativedynamicsreview}. Nonetheless, many bulk properties of the Ising model, such as the critical exponents at the CQT point, are preserved by the mapping; in fact, these phase transition belongs
to the same universality class of the 1D quantum Ising model.\\


The quantum Ising model with a transverse field is one of the most common theoretical laboratories where fundamental issues on quantum phase transition can be addressed, given our deep knowledge of the quantum correlations~\cite{S99}. The model is characterized by a $\mathbb{Z}_2$ global symmetry under spin reflection along the longitudinal axis. In Eq.~\eqref{Hkitaev}, this symmetry is implemented by the transformation that maps $\hat{c}^{(\dagger)}_x\to -\hat{c}^{(\dagger)}_x$. At zero temperature, the ground state experiences a CQT at $\mu_c=-2$ and the $\mathbb{Z}_2$ symmetry is then spontaneously broken. The critical point separates a paramagnetic phase ($\abs{\mu}<\abs{\mu_c}$), where correlation functions are exponentially dumped, from an ordered phase ($\abs{\mu}<\abs{\mu_c}$), where correlation functions are instead long-range ordered. Close to the critical point, the correlation length diverges as $\xi\sim\abs{\mu-\mu_c}^{-\nu}$, where $\nu=1/y_g=1$ for Ising transitions. The gap $\Delta$, which describes the energy difference between the first excited state and the ground state, vanishes instead as $\Delta\sim\xi^{-z}$ with $z=1$. 



