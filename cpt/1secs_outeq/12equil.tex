\section{Equilibrium Quantum Transitions}

\subsection{Continuous Quantum Transition (CQT)}

In the thermodynamic limit, i.e. infinite volume limit, same specific models could undergo a continuous phase transition. With the word phase, we intend the physical properties of the ground stare associated with the Hamiltonian and characterized by the values of the Hamiltonian parameters. If this phase changes driving one of this parameter, we are in the presence of a quantum transition and the point, in which that happens, is called transition point.

In the particular case of the CQT, the quantum ground-state properties are continuous close to the transition point, called also critical point. 
In this special point, the system develops a long-distance correlations and its microscopic behavior becomes negligible.

Independent from the local details, the global properties determine a notable universal critical behavior in which we collect different physical systems in universality class.
In terms of the Renormalization Group (RG) theory, the critical point and its universal behavior ase associated the fixed points of the RG flux. Hence, if we call $b$ the characteristic unit length of the system and if we define the RG transformation as the parameter rescaling respect $b$, the physical observables satisfy general scaling law unchanged along all the RG flux.

From a physics point of view, we can interpret the factor $b$ like the spatial correlation length $\xi$ which diverges approching the critical point.
According to the renormalization group (RG)
theory of critical phenomena, these global properties may be the spatial dimension-
ality, the nature of the order parameter, the symmetry and the symmetry-breaking
pattern.

Moreover, at CQTs the systems develop an equilibrium and dynamic scaling be-
havior in the thermodynamic limit and their quantum functions satisfy scaling
power laws characterized by universal critical exponents.

\subsection{First Order Quantum Transition (FOQT)}

Quasi-degenerate vacua naturally arise in the context of quantum phase transitions, after a spontaneous symmetry breaking. Their behavior and coexistence in the non-critical regime is governed by a first-order quantum transition.
FOQTs are associated with many important out-of-equilibrium effects s, including nucleations and metastability \cite{binder1987theory,bray2002theory}, coarsening \cite{chandran2012kibble}, and anomalous dependence on the boundary conditions \cite{pelissetto2020scaling,panagopoulos2018dynamic,campostrini2015quantum,pelissetto2018finite,rossini2018ground}.

\subsection{Models}



