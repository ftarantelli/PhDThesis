\chapter{Out-of-equilibrium dynamics}
\label{chp_out}

%\input{cpt/1secs_outeq/11intro}
\section{Introduction}

% This chapter is devoted to introduce the basis and the notation associated with the analysis
% of quantum system in the presence of processes which generate the out-of-equilibrium 
% dynamics. In particular, we start from system close their equilibrium quantum transition
% point and, then, we turn on two different time perturbations:

In this chapter, we briefly introduce the main concepts which are the starting point
of all the research statement. We will described the definition of transitions for
many-body systems and some processes that drive out from this equilibrium, like:

\begin{itemize}
	\item
		the unitary round-trip processes whose time-evolution is associated with
		an hermitian Hamiltonian;
	\item
		the dissipation processes in which we put our quantum system in contact with
		external baths.
\end{itemize}

We will always consider models which undergo a continuous quantum transition (CQT)
and/or a first order transition (FOT). 

All these concepts will be explored with more details in the future chapters of the 
work. The aim is to understand the behavior of a quantum system and its dynamics
out of the equilibrium regime. We will use the Out-of-equilibrium
Finite Size Scaling (OFSS) theory to discover the properties of these complex
systems.


% in the case of a non-unitary dynamics associated with a weak dissipation interaction 
% between the quantum system and the external environmental, the master equation for the
% system density matrix $\rho$ in the secular and Born-Markov approximations (for further 
% details see \cite[BP-openquantumsystembook, TV21]):
% \be{evol_rho}
% 	\frac{d\rho}{dt} = - i\bigr[ H, \rho \bigr] + \mathbb{D}(\rho) \pt
% \ee





\section{Equilibrium Quantum Transitions}

\subsection{Continuous Quantum Transition (CQT)}
\label{CQTeq}

In the thermodynamic limit, i.e. infinite volume limit, same specific models could undergo a continuous phase transition. With the word phase, we intend the physical properties of the ground stare associated with the Hamiltonian and characterized by the values of the Hamiltonian parameters. If this phase changes driving one of this parameter, we are in the presence of a quantum transition and the point, in which that happens, is called transition point.

In the particular case of the CQT, the quantum ground-state properties are continuous close to the transition point, called also critical point. 
In this special point, the system develops a long-distance correlations and its microscopic behavior becomes negligible.

Independent from the local details, the global properties determine a notable universal critical behavior in which we collect different physical systems in universality class.
In terms of the Renormalization Group (RG) theory, the critical point and its universal behavior ase associated the fixed points of the RG flux. Hence, if we call $b$ the characteristic unit length of the system and if we define the RG transformation as the parameter rescaling respect $b$, the physical observables satisfy general scaling law unchanged along all the RG flux \cite{S99, C-1996-ScalingandRG}.

From a physics point of view, we can interpret the factor $b$ like the spatial correlation length $\xi$ which diverges approching the critical point.
According to the renormalization group (RG)
theory of critical phenomena, these global properties may be the spatial dimension-
ality, the nature of the order parameter, the symmetry and the symmetry-breaking
pattern~\cite{PV2002}.

Moreover, at CQTs the systems develop an equilibrium and dynamic scaling be-
havior in the thermodynamic limit and their quantum functions satisfy scaling
power laws characterized by universal critical exponents~\cite{CV2014}.

\subsection{First Order Quantum Transition (FOQT)}
\label{FOQTeq}

Quasi-degenerate vacua naturally arise in the context of quantum phase transitions, after a spontaneous symmetry breaking. Their behavior and coexistence in the non-critical regime is governed by a first-order quantum transition.

It is characterized by the crossings of the lowest-energy states in the infinite-volume limit and in the absence of conservation laws \cite{plissetto2023scaling}.
Instead, in a finite system, the energy gap among these states remain different from zero, giving rise to the phenomenon of avoided level crossing.

FOQTs are associated with many important out-of-equilibrium effects s, including nucleations and metastability \cite{binder1987theory,bray2002theory}, coarsening \cite{chandran2012kibble}, and anomalous dependence on the boundary conditions \cite{pelissetto2020scaling,panagopoulos2018dynamic,campostrini2015quantum,pelissetto2018finite,rossini2018ground}.

\subsection{Renormalization Group (RG) Theory}

The Renormalization Group (RG) is a powerful analytical framework with broad applications across various fields of physics. This brief review focuses on its application to critical phenomena.

The strength of the RG approach lies in its ability to reveal a new symmetry at the critical point $ P = P_c $, which is not apparent in the original Hamiltonian. This symmetry arises due to the divergence of the correlation length $ \xi $ at $ P = P_c $. As $ \xi $ approaches infinity, the system exhibits invariance under scaling transformations where the coordinates are rescaled as 

\ba{scale_transformation}
x \rightarrow x' = \frac{x}{b}
\ea

with $ b $ being the rescaling factor. This implies that the fundamental structure of correlations remains unchanged when viewed at different length scales, which is a characteristic of scale invariance. For instance, the two-point correlation function 

\ba{scale_correlation}
F(x) \sim x^{2-D}
\ea

for $ x \ll \xi $ is rescaled by a pre-factor under such transformations.


It's important to note that lattice models, while having a characteristic length scale equal to the lattice spacing $ a $, exhibit this scaling invariance only at length scales much larger than $ a $ and precisely at the critical point.


The role of RG is similar to other familiar symmetries, such as rotational invariance. Just as rotational symmetry classifies observables into scalars, vectors, tensors, etc., scale invariance classifies observables by their scaling dimensions.

Approaching the critical point (or fixed point), the system's behavior is governed by universal scaling laws that depend on the relevant parameters. These laws imply that the parameters must be rescaled by a common factor $ b $, and any irrelevant parameter with a negative critical exponent can be neglected. Consequently, at the critical point, where the correlation length is the relevant length scale, physical observables must remain invariant under such rescaling. For example, the two-point correlation function $ F(\bm x, \bm y, \bar{r}, h) $, which captures the critical behavior's non-analyticity and universal features, obeys the general scaling law:

\ba{scaling_correlation}
F(\bm x, \bm y, \bar{r}, h) \sim b^{- y_c} \mathscr{F}\left(\frac{\bm x}{b}, \frac{\bm y}{b}, \bar{r}b^{y_r}, hb^{y_h}\right)
\ea

where $ y_c $ is the scaling exponent associated with the operators in $ F(\bm x, \bm y, \bar{r}, h) $, $ y_r = \frac{1}{\nu} $ is the renormalization group dimension (or critical exponent) of the relevant parameter $ \bar{r} $, and $ y_h $ is the renormalization group dimension of the parameter $ h $. This scaling law suggests that near the critical point, the system effectively averages over large volumes, making microscopic details less significant.

$ $\\

In quantum systems, the interplay between energy scales $ \Delta $ and thermal energy $ T $ determines the scaling behavior. When $ \Delta \ll T $, quantum effects dominate, and as $ \bar{r} \rightarrow 0 $, the energy gap diminishes, leading to a phase transition through spontaneous symmetry breaking. Conversely, when $ \Delta \gg T $, a classical description suffices for order parameter fluctuations.

In the quantum critical region, where both thermal and quantum fluctuations are significant, critical behavior can be approached either by $ r \rightarrow r_c $ at $ T = 0 $ or $ T \rightarrow 0 $ at $ r = r_c $. The scaling laws governing physical observables in these scenarios are often related, and for small temperatures $ T $ near the critical point, the form of the scaling law reflects these dependencies:
\begin{equation}
	\label{scalinglawT}
	F(\bm x, \bm y;\,\bar r,\,h,\,T) \sim b^{-y_c} \, \mathscr{F}\biggl( \frac{\bm x}{b} , \frac{\bm y}{b}, \,\bar r \,b^{1/\nu}, \, h\,b^{y_h} , \,T\,b^z \biggl) \,\, ,
\end{equation}
where we introduce the scaling parameter $\,T\,b^z \,$ to consider the explicit dependence of the quantum system from the low temperature $T$ when $\bar r=0\,$~\cite{S99}.


\subsection{Finite Size Scaling (FSS)}

With the quantum-to-classical mapping, we can associate the quantum system on a spatial volume $L^d$ with a classical one on a volume $L^d \cdot L_t$. In this way, at QTs the scaling of the temperature at the critical point in d-dimension is analogous to a FSS in the corresponding (d+1)-dimensional classical system. In fact, the inverse temperature in the quantum system corresponds to the system size in an imaginary time direction.  In this viewpoint, in Eq. (\ref{scalinglawT}) the scaling term $T\,b^z\,$ of the quantum system comes from the FSS of the corresponding classical system with a finite length $L_t \sim T^{-\,\frac{1}{z}}\,$. The finiteness of $L_t$ or another dimensional length  modifies the critical behavior taking two different form. One, it can destroy the transition entirely, so that the only critical point is at $\,L_t \to +\infty\,$. Latter, the modification is such that the transition persists to finite $L_t$, but crosses over to a different universality class. \\
For example, we consider a prototype system with at least one finite size length $L\,$ and $\xi(\bar r, \, h ;\,L \to +\infty)\,$ the correlation length of the corresponding system when $\,L\to +\infty\,$. If $\,L\gg  \xi(\bar r, \, h ;\,L \to +\infty)\,$, then we are in the thermodynamic limit where, obviously, no significant finite size effects are observed. \\
Instead, for $\, L \sim \xi(\bar r, \, h ;\,\infty)\,$, the system size cuts off the long-distance correlations and the critical behavior around the critical point is described by a FSS framework. For large sizes, the FSS behavior is universal for all the systems whose CQT belongs to the same universality class. In particular, the FSS provides the asymptotic scaling behavior only if we fix the ratio $\xi/L$ taking the limit $L \to +\infty\,$. Since in the computational results the system size and the range of parameter values are often small, in the FSS framework we can find sizable scaling corrections which may not allow to estimate the critical parameters. Thus, we must control the corrections to the asymptotic behavior to accurate estimate the values of these critical parameters.\\


In the quantum-to-classical mapping, we relate a quantum system defined within a spatial volume $L^d$ to a classical system occupying a volume $L^d \cdot L_t$. This correspondence allows us to draw parallels between the scaling behavior of temperature at the critical point in a d-dimensional quantum system and finite size scaling (FSS) in the corresponding (d+1)-dimensional classical system. Specifically, the inverse temperature in the quantum framework corresponds to the system size in the imaginary time direction.

In this perspective, the scaling term $T\,b^z\,$ in Eq. (\ref{scalinglawT}) arises from the FSS of the classical system, which features a finite length $L_t \sim T^{-\,\frac{1}{z}}\,$. The finiteness of $L_t$, or any other characteristic length scale, can modify the critical behavior in two significant ways:

\begin{enumerate}
	\item It may entirely suppress the phase transition, resulting in a critical point only at $\,L_t \to +\infty\,$

	\item Alternatively, the transition may persist at finite $L_t$ but shift to a different universality class.
\end{enumerate}

  
To illustrate, consider a prototype system characterized by at least one finite size length $L$ and its correlation length $\xi(\bar r, \, h ;\,L \to +\infty)\,$  in the limit $L \to +\infty\,$. When $\,L\gg  \xi(\bar r, \, h ;\,L \to +\infty)\,$, we are in the thermodynamic limit, and no significant finite size effects are observed.

Conversely, when $\, L \sim \xi(\bar r, \, h ;\,\infty)\,$, the system size truncates long-range correlations, necessitating a FSS framework to accurately describe the critical behavior near the critical point. For large system sizes, the FSS behavior becomes universal across systems belonging to the same universality class. Importantly, the FSS provides asymptotic scaling behavior only when the ratio $\xi/L$ is fixed while taking the limit $L \to +\infty\,$. Given that computational studies often involve limited system sizes and parameter ranges, substantial scaling corrections can emerge in the FSS framework, complicating the estimation of critical parameters.

Since the leading term of the correlation length is $\xi \sim L\,$, we can assume $b=L$. Consequently, the asymptotic FSS behavior of the correlation length is expressed as:
\begin{equation}
\label{xifsscqt1}
\xi(\bar r, \, h ;\,L) \sim
L \cdot X_L\bigl( 
\bar r\,L^{1/\nu} ,\,h\,L^{y_h} \,    \bigl) \,\, ,
\end{equation}
where $y_h$ is the critical exponent of the Hamiltonian parameter $h\,$. This relationship stresses the critical interplay between system size and correlation length in the context of quantum phase transitions.\\
Moreover, from $\Delta \sim L^{-z}\,$, the low-energy gap $\Delta$ is expected to show the asymptotic FSS behavior:
\begin{equation}
\label{deltafsscqt}
\Delta _L(\bar r, \, h) \equiv \Delta _L(\bar r, \, h;\,L) \sim
L^{-z} \, D_L\bigl( 
\bar r\,L^{1/\nu} ,\,h\,L^{y_h} \,    \bigl) \,\, .
\end{equation}
$ $\\
For the correlation function of the order-parameter field $\phi(\bm x, t)\,$:
\begin{equation}
\label{corr-ord}
F(\bm x, \bm y) = \braket{\phi(\bm x,t)\,\phi(\bm y,t)}
\,\, ,
\end{equation}
the leading scaling behavior in the condition $h=0$, for $L\to +\infty\,$, with $\bm x- \bm y/L$ fixed,  is equal to:
\begin{equation}
\label{scal-corr}
F(\bm x, \bm y, \bar r;L) \approx u_l^{d+z-2+\eta} \, \mathscr{F}(u_l\bm x ,u_l\bm y, \bar r\,u_l^{-1/\nu})
\,\, ,
\end{equation}
where $\eta$ is known as the anomalous dimension of the field $\phi\,$ and $u_l$ has an expansion for $L\to +\infty\,$: $u_l = 1/L + b/L^2 + \dots = 1/L_e + O(L_e^{-3})$ with $L_e = L-b\,$. In particular, the non-universal constant $b$ is equal to $0$ for periodic or aperiodic BC with short-range interactions because the translation-invariant couplings are only functions of the couplings of the original lattice and are independent of $L$ along the RG flux. If, now, $\phi$ is a linear combination of the operators ${\cal O}_{h,i}$ $\phi = \sum _i a_i\,{\cal O}_{h,i}\,$, 
then the Eq. (\ref{scal-corr}) provides the contribution of the leading operator ${\cal O}_h$ because its RG dimension is given by: $\,\,\, d + z - y_h = \frac{1}{2}\left(d+z-2+\eta\right)\,$.\\
In the open BC case, we must consider in Eq. (\ref{scal-corr}) the subleading terms arising from the irrelevant scaling fields $v_i\,u_l^{-y_i}\,$ and from the leading boundary operators $\widetilde{v}_i\,u_l^{-\widetilde{y}_i}\,$. Hence, the equation (\ref{scal-corr}) becomes:
\begin{equation}%expected for open BC
\label{scal-corr1}
F(\bm x, \bm y, \bar r;L) \approx u_l^{d+z-2+\eta} \,\mathscr{F}\Bigl(u_l\bm x ,u_l\bm y, \bar r\,u_l^{-1/\nu}, \bigl\{v_i\,u_l^{-y_i}\bigl\},  \bigl\{\widetilde{v}_i\,u_l^{-\widetilde{y}_i}\bigl\}  \Bigl)
\,\, .
\end{equation}


\begin{comment}

	\subsection{Models}
	\label{Models}

	\subsubsection{Quantum Ising}
	\label{QIsing}

	As a first toy model for the study of quantum phase transitions, we consider the quantum 1D Ising Model whose Hamiltonian is given by:

	\be{HIsing}
		H(g, h) = - \sum_{x=1}^{L-1} \sigma^{(1)}_x \sigma^{(1)}_{x+1}
		- h \sum_{x=1}^L \sigma^{(1)}_x - g \sum_{x=1}^L \sigma^{(3)}_x \pc
	\ee

	where $L$ is the system size and $\sigma_x^{(k)}$ are the Pauli matrices on the 
	$x^{\rm th}$ site.\\

	This system develops a quantum critical behavior at $g = g_c = 1$
	and $h=0$, belonging to the 2D Classical Ising universality class \cite{S99}. Instead, 
	when $h$ is different from zero, the lowest states energy gap is not vanish.

	Along the RG flux, the relevant parameters associated with the RG perturbations at the 
	fixed point are $r = g-g_c$ and $h$. Their RG dimensions are respectively $y_r=1/\nu=1$
	and $y_h = 15/8$, so that the length scale $\xi$ of the critical modes behaves as
	$\xi \sim \abs{g-g_c}^{-1/y_r}$ for $h=0$, and $\xi \sim \abs{h}^{-1/y_h}$ for $g=g_c$.
	The dynamic exponent $z$ associated with the vanish critical gap $\Delta \sim \xi^{-z}$
	at the transition point, is given by $z=1$. The order parameter field, which distinguishes 
	the two phases and is associated with the longitudinal operators $\sigma_x^{(1)}$, has a
	RG dimension equal to $y_l = d+z-y_h=1/8$, while that associated with the transverse
	operator $\sigma_x^{(3)}$ is equal to $y_t = d+z-y_r = 1$.\\

	In the ferromagnetic phase $g<1$, the model undergoes a FOQT at $h=0$. Across this point,
	the system remains non-critical and displays exponential decay of the correlation 
	functions.

	For $h=0$, the model presents a level-crossing of the two lowest-energy state in the
	infinite volume limit, where the energy gap closes exponentially for $L\to \infty$, e.g.
	for open boundary condition \cite{cabrera1987role}:
	\be{isingfoqtgap}
		\Delta(g,L) = 2 g^L(1-g^2) \Bigr[ 1+ {\cal O}(g^{2L}) \Bigr] \pt
	\ee
	In the limit of small longitudinal magnetic field, i.e. $\abs{h \ll 1}$, the system
	presents a Zeeman-like gap in energy between the 2 lowest states which introduces another
	symmetry-breaking of the degeneracy. We can express an approximation of this gap using
	standard perturbation theory in $h$ \cite{campostrini2014finite}:
	\be{isingperthgap}
		{\cal E}(g,h\to 0,L) \simeq 2 h \sum_{j=1}^{L}
		\abs{\braket{\sigma_j^{(3)}}} \simeq 2hLM_0(g) \cm
	\ee
	with $M_0 = (1-g^2)^{1/8}$ the approximated longitudinal magnetization in the constrains
	$h = 0$ and $L\to \infty$.



	\subsubsection{Kitaev chain}
	\label{kitaev}

	Now, we introduce the following lattice model, called Kitaev model \cite{Kitaev-2001},
	whose Hamiltonian describes the fermions interaction with the lattice and is given by:
	\ba{Hkitaev}
		\hat{H}&=-\sum_{x=1}^{L-1} (\hat{c}^{\dagger}_{x}\hat{c}_{x+1} 
		+ \hat{c}^\dagger_{x}\hat{c}^\dagger_{x+1}+{\rm h.c}.)
		- \mu\sum_{x=1}^{L} \hat{n}_x\,,
		%\label{Hkitaev}
	\ea
	where $\hat{n}_x\equiv\hat{c}^\dagger_x\hat{c}_x$ is the number operator on the site $x$, and the operators $\hat{c}_x, \hat{c}^\dagger_x$ satisfy the canonical anticommutation relations, thus $\{\hat{c}_x, \hat{c}_y\}=\{\hat{c}^\dagger_x, \hat{c}^\dagger_y\}=0$ and $\{\hat{c}_x, \hat{c}^\dagger_y\}=\delta_{xy}$. Applying the Jordan-Wigner transformation~\cite{S99}, the Kitaev ring can be exactly mapped into a quantum Ising chain with a transverse field~\cite{PF70}. We point out that the transformation does not preserve also the same boundary conditions, so attention should be paid when recasting Eq.~\eqref{Hkitaev} in its bosonic counterpart~\cite{RV-2021-coherentanddissipativedynamicsreview}. Nonetheless, many bulk properties of the Ising model, such as the critical exponents at the CQT point, are preserved by the mapping; in fact, these phase transition belongs
	to the same universality class of the 1D quantum Ising model.\\


	The quantum Ising model with a transverse field is one of the most common theoretical laboratories where fundamental issues on quantum phase transition can be addressed, given our deep knowledge of the quantum correlations~\cite{S99}. The model is characterized by a $\mathbb{Z}_2$ global symmetry under spin reflection along the longitudinal axis. In Eq.~\eqref{Hkitaev}, this symmetry is implemented by the transformation that maps $\hat{c}^{(\dagger)}_x\to -\hat{c}^{(\dagger)}_x$. At zero temperature, the ground state experiences a CQT at $\mu_c=-2$ and the $\mathbb{Z}_2$ symmetry is then spontaneously broken. The critical point separates a paramagnetic phase ($\abs{\mu}<\abs{\mu_c}$), where correlation functions are exponentially dumped, from an ordered phase ($\abs{\mu}<\abs{\mu_c}$), where correlation functions are instead long-range ordered. Close to the critical point, the correlation length diverges as $\xi\sim\abs{\mu-\mu_c}^{-\nu}$, where $\nu=1/y_g=1$ for Ising transitions. The gap $\Delta$, which describes the energy difference between the first excited state and the ground state, vanishes instead as $\Delta\sim\xi^{-z}$ with $z=1$. 
\end{comment}




\section{Unitary time-evolution}



%\subsection{Quench}
\subsection{Dynamic Scaling Theory}

In the quench protocol, we define a family of Hamiltonians of type:
\ba{Hquench}
H(\bar \mu) = H_o + \bar\mu P \cm
\ea
where in this case the scaling variable $\bar \mu$ tunes the strength of the perturbation
$P$ such that $\Bigr[ H_o , P] \neq 0$ and $H_o$ is the unperturbed Hamiltonian 
whose parameters assume their critical values.

In this quench protocol, at $t = t_0$ the system starts in the ground state
of the Hamiltonian associated with an initial value $\bar \mu_i$ . Then, at time $t>0$, we
suddenly change the coupling from $\bar \mu$ to $\bar \mu _i$ and we follow the 
corresponding evolution of the system, described by the Schro\"odinger equation
in the form of the Von-Neumann equation:
\ba{eqschrodinger}
\partial_t \rho(t) = -i \Bigr[ \hat H(\bar \mu), \rho(t) \Bigr] \qquad \rho(t=0) = \rho_0 \,\,,
\ea
where $\rho(t)$ is the system density matrix at time $t\,$.

To express a possible scaling law, we define a further scaling variable associated
with the time:
\be{theta_din}
\theta = t \Delta \cm
\ee
which is obtained by recalling that the inverse energy difference of the lowest states
is proportional to the relevant time scale of the critical modes.

$ $\\

In the finite size system $\Delta _L \sim L^{-z}\,$, therefore in the infinite volume limit $L\to +\infty$, the out of equilibrium dynamics show a FSS behavior in which the scaling variable $\theta$ enters in the scaling functions. Hence, for this asymptotic dynamic scaling, we can write the following FSS law related to the correlation functions $F\,$: 
\begin{equation}  
\label{dfss-C}
F(\bm x, \bm y, \bar r;L) \approx u_l^{d+z-2+\eta} \,\mathscr{F}\Bigl(u_l\bm x ,u_l\bm y, \bar r\,u_l^{-1/\nu}, \bigl\{v_i\,u_l^{-y_i}\bigl\},  \bigl\{\widetilde{v}_i\,u_l^{-\widetilde{y}_i}\bigl\}, \theta  \Bigl) \,\,.
\end{equation}

\subsection{Kibble-Zurek mechanism}

The Kibble-Zurek(KZ) mechanism is related to the amount of final defects after slow
passages through continuous transition, from disordered to the ordered phase 
\cite{kibble1976topology, kibble1980some, zurek1985cosmological, zurek1996cosmological, 
zurek2005dynamics}. This type of out-of-equilibrium process is several studied both
analytically-numericallly \cite{dziarmaga2010dynamics, PSSV-2011-noneqcoll,
chandran2012kibble, rossini2021coherent} both experimentally \cite{weiler2008spontaneous,
ulm2013observation}.

The large-scale modes, associated with the changes of the transition tuning parameter, are
insufficient to equilibrate the long-distance critical correlations. Even in the large
time variation regime, the out-of-equilibrium dynamics grows in the thermodynamic limit.
In other words, when the system evolves, starting from an equilibrium state, the
time-evolution is different from an adiabatic dynamics and the system does not pass 
through equilibrium states.

In this scenario, the out-of-equilibrium regime is always describable in terms of the RG
framework and the equilibrium scaling behaviors can be related by the quantum to classical
mapping \cite{rossini2021coherent, S99}.




\section{Time-evolution in Open Quantum Systems}


\subsection{Lindblad framework}

To model the weak interaction between the previous quantum model and the surrounding
environment, we consider local external baths each in contact with only site of the system
chain. We work under the Born-Markov and secular approximations, so dissipators 
can be effectively modeled employing Lindblad quantum jump operators $\hat L_x\,$. 
In this limit, the time evolution of the density matrix can be described by Markovian
master equations in the Lindblad form as \cite{BP-openquantumsystembook, TV-2021-dissipativeboundaries, TV-2022-localizedparticleloss}:
\ba{eqlindblad}
	\frac{d\rho}{dt} = {\cal L}[\rho] = 
		-i \Bigr[ H, \rho \Bigr] + \mathbb{D}[\rho] \pc
\ea

where ${\cal L}$ is the Liouville superoperator, and $\mathbb{D}$ is the corresponding 
dissipation term, whose strength is regulated by the coupling $w$:
\ba{dissipator}
	\mathbb{D}[\rho] = & w \sum_{x \in {\cal I}} 
		\mathbb{D}_x[\rho] \cm \\
	\mathbb{D}_x[\rho] = &
		\hat L_x \rho \hat L_x^\dagger - 
		\frac{1}{2} \Bigl\{ \rho, \hat L^\dagger_x \hat L_x \Bigl\} \pc
\ea
where we indicate with ${\cal I}$ the set of the external baths in contact with the
quantum system.

\begin{comment}

\subsection{Thermal bath}

If we are interested to achieve an equilibrium Gibbs state at some temperature $T$, we
can consider the following modelization of interaction with a thermal bath within 
the Lindblad master equation \eqref{eqlindblad}.
Let us take a  quantum models described by quadratic Hamiltonians \cite{dr2021self}, 
such as that of the fermionic Kitaev model \eqref{Hkitaev}. This provides a relatively 
simple modelization of a thermal bath leading to thermalization in the large-
time limit of the corresponding Lindblad master equation
for the density matrix of the system.

The Kitaev Hamiltonian with open boundary conditions can be diagonalized 
in the Nambu field space by a Bogoliubov transformation \cite{dr2021self, PF70, bla86},
so that we can rewrite it as:
\ba{diagHKitaev}
	\hat H_K(\omega)  = \sum_{k=1} ^L \omega_k \hat b_k^\dagger \hat b_k  \cm
\ea
where $\omega_k$ are values of the spectrum of the Bogoliubov eigenoperators $\hat b_k$
(we are neglecting an irrelevant constant term). Note that both $\omega_k$ and 
$\hat b_k$ depend on the Hamiltonian parameter $\mu\,$. The relation between the
fermionic operators $\hat c_x$ and the Bogoliubov eigenoperators 
$\hat b_k$ can be generally written as \cite{dr2021self, PF70, bla86}:
\ba{basisfrombtoc}
	\hat c_x = \sum_{k=1}^L A_{xk} \hat b_k + B_{xk} \hat b^\dagger _k \cm
\ea
where $A$ and $B$ are appropriate $L{\rm x}L$ matrices depending on $\mu\,$.
Following Refs. \cite{dr2021self, CPR-2022-otto_engine}, we write the dissipator
$\mathbb{D}_T[\rho]$ in the Lindblad master equation \eqref{eqlindblad} in terms of
the Bogoliubov eigenoperators as:
\ba{thermaldissipator}
	\mathbb{D}_T[\rho] = & \gamma \sum_k \bigr[ 1 - f(\omega_k, T) \bigr] 
		\bigl( 2 \hat b_k \rho \hat b_k^\dagger -
			\{ \hat b_k^\dagger \hat b_k, \rho \} \bigl) + \\
		& + \gamma \sum _k f(\omega_k, T) \bigl( 2 \hat b_k^\dagger \rho \hat b_k -
                        \{ \hat b_k \hat b_k^\dagger, \rho \} \bigl) \cm
\ea
where:
\be{thermaldensityfunction}
	f(\omega_k , T ) = \bigl( 1 + e^{\omega_k/T} \bigl)^{-1} \pt
\ee
When using this dissipator term, the Lindblad master equation 
\eqref{eqlindblad} ensures the asymptotic large-time thermalization \cite{dr2021self}.
Therefore,
\ba{thermalstationarystate}
	\lim_{t \to \infty} \rho(t) & = \rho_t(\omega, T ) \cm \\
	\rho_t (w, T ) &  = \sum_n e^{-E_n(\omega)/T} \ket{\phi_n, \omega} 
					\bra{\phi_n, \omega} \cm
\ea
where $\rho_t(\omega,T)$ is the density matrix representing the
thermal state, $E_n (\omega)$ and $\ket{\phi_n, \omega}$ are the eigenvalues
and eigenstates of $\hat H(\omega)\,$. The asymptotic approach to
the thermal distribution is controlled by the decay-rate parameter $\gamma$.

\end{comment}

\subsection{Liouvillian gap}
\label{subsec_liouvilliangap}

Let us consider the following equation:
\ba{eigencalL}
	\widetilde{\cal L}[\widetilde \rho_i] = \lambda_i \widetilde \rho_i \cm
	\qquad \lambda_i \in \numberset{C} \pc
\ea
where $\widetilde {\cal L}$ is the (non-hermitian) Lindblad superoperator after the 
Choi-Jamiolkowski isomorphism 
\cite{BP-openquantumsystembook, VZ-2004-superoperatorvidal}, and $\widetilde \rho_i$ is the
density matrix eigen-operator associated with the complex eigenvalue $\lambda_i$.
In a few words, the transformation we have mentioned sends the density matrix $\rho$ to 
$\widetilde \rho$ through the mapping:
\be{vecrho}
	\rho_{ij} \ket i \bra j  \longrightarrow  \widetilde \rho_{ij} \ket i \ket j \pt
\ee
Therefore, the vectorized $\widetilde \rho$ lives in a $4^L$-dimensional Hilbert space.
In this basis, the action of $\widetilde{\cal L}$ on 
$\widetilde \rho$ can be written as follows:
\ba{vecteqlindblad}
	\widetilde{\mathcal{L}} =& -i \big(\hat{H} \otimes \hat{\mathbb{I}} 
		- \hat{\mathbb{I}}\otimes \hat{H}^t \big) + 
		w\sum_{x \in {\cal I}}\hat{L}_{x}\otimes \hat{L}^*_{x}\\
	&-\frac{w}{2}\sum_{x \in {\cal I}}\big(\hat{L}^{\dagger}_{x}\hat{L}_{x}
		\otimes\hat{\mathbb{I}}+\hat{\mathbb{I}}
			\otimes\hat{L}^t_{x}\hat{L}^*_{x}\big) \pt
\ea
It can be shown that all eigenvalues of $\widetilde{\cal L}$ satisfy 
$ {\rm Re}{\lambda_i } \le 0$ \cite{BP-openquantumsystembook}. The zero mode of the above 
operator represents the steady-state solution, namely, the NESS of the system. 
If $\hat L_x$ is not hermitian, the density matrix corresponding to the steady-state 
solution is not proportional to the identity matrix \cite{KS-2020-boundarydephasing}. 
We focus on the Liouvillian gap $\Delta_{\lambda}$, which is the non-vanishing eigenvalue
of ${\cal L}$ with the smallest real part:
\ba{Lioulliangap}
\Delta_{\lambda} = - \max_{i} {\rm Re}{\lambda_i } \pt
\ea
This quantity controls the typical relaxation time of the longest-living 
eigenmode differing from the NESS.

