\section{Summary}

\rev{
When we keep $b$ fixed, the gap $\Delta_\lambda$ is always finite and depends linearly on the dissipation strength $w$. Nonetheless, two different regimes emerge for systems of finite size. In the small $w$ region, the gap is given by $\Delta_\lambda=w/(2b)$, whereas, at large $w$ and sufficiently large $b$, it behaves as $\Delta_\lambda=w C_\mu/b^3$. The last equation always controls the gap in the large-size limit and is our starting point to deduce the scaling of such a quantity when $b\propto L$. It is worth mentioning that we also put forward a scaling regime for $L\Delta_\lambda$ as a function of $wL$, which ties together the two different regimes outlined in a smooth manner. On the other hand, when we keep the number of dissipators $n$ fixed, the gap vanishes as $\sim L^{-3}$ at large $L$. Addressing the structure of the gap at small $w$, we find a scaling regime for $L^2\Delta_\lambda$ in terms of $wL$, which is closely related to the presence of a non-uniform convergence of $L^3\Delta_\lambda$ in the limit 
$w\to0^+$.\\$ $\\
We develop a dynamic FSS regime at CQTs to describe the time evolution of the Kitaev model under investigation. At fixed $b$, our results extend the FSS theory of Ref.~\cite{NRV-2019-competingdissipativeandcoherent} to the cases with $b>1$. As a working hypothesis, we suppose that the scaling variable associated with the relevant coupling $w$ is $\gamma_b=wL^{z}/b$. Our numerical results for the two-point correlation functions fully support this ansatz. When the number of dissipators $n$ is fixed, the FSS theory outlined at fixed $b$ generalizes straightforwardly after replacing $\gamma_b$ with $W=wL^{z-1}$. \\$ $\\
%In the last section, we analyze the interplay between the Liouvillian gap $\Delta_\lambda$ and the gap related to the Kitaev ring $\Delta$ in the FSS limit. In particular, we take into account the short- and long-time regimes, focusing on how they join together in the FSS limit. When $b$ is fixed, we observe that the link between the two regimes is smooth, whereas, at fixed $n$, the two regions can be easily distinguished given the presence of different power-law scalings for the gaps $\Delta$ and $\Delta_\lambda$. 
}

