\section{Summary}

\rev{
    We studied the out-of-equilibrium dynamics of fermionic gases confined within finite spatial regions and subject to localized particle-loss dissipation. These systems include fermions constrained by hard walls or trapped in a spatially dependent potential, such as a harmonic trap, which are common setups in cold-atom experiments. The particle loss, modeled as a localized dissipative interaction, was investigated using the Lindblad master equation for the density matrix, starting from equilibrium conditions.
    We considered one-dimensional spinless fermionic lattice gases with particle loss localized at a specific site, such as the center of the system. Our study revealed several dynamic scaling regimes that depend on the interplay between the size \( L \) of the system and time. Notable differences were observed between hard-wall and harmonic-trap confinements.\\
    At large times, the number of particles \( N(t) \) stabilizes, with half of the initial particles remaining for even \( N_0 \), and slightly fewer for odd \( N_0 \). The particle density \( n_x(t) \) also stabilizes, vanishing at the dissipation site.\\
    The approach to the stationary state is controlled by the Liouvillian gap, which scales as \( L^{-3} \). This implies that the time scale for reaching the stationary state behaves as \( t_a \sim L^3 \).\\
    Different scaling behaviors appear depending on the initial conditions:
    \begin{itemize}
        \item For a fixed particle number \( N_0 \), intermediate dynamics scale as \( t/L^2 \), related to the Hamiltonian gap.
        \item For a constant ratio \( N_0/L \), intermediate dynamics scale as \( t/L \), reflecting the thermodynamic limit. 
    \end{itemize}
    Similar to the hard-wall case, the particle number stabilizes at large times, but the density \( n_x(t) \) continues to oscillate over time at non-dissipative sites, without reaching a completely static state.\\
    The time to stabilize the particle number is proportional to the trap size \( t_a \sim L_t \), and scaling functions depend on \( t/L_t \).\\
    A central localized loss leads to two distinct dynamic regimes. Initially, particle number decreases due to dissipation, but when the density at the dissipation site vanishes, the remaining particles are conserved. This behavior is linked to the non-interacting nature of fermionic gases, where modes vanishing at the dissipation site are unaffected.
    \\$ $\\
    %Now, to establish a "bridge" between the local and uniform dissipation processes, we consider the interessing case of the sunburst model.
    To establish a connection between local and uniform dissipation processes, we analyze the intriguing case of the sunburst model.
    In particular, we investigated the dynamics of a (1+1)-dimensional Kitaev ring coupled to an environment through \( n \) particle-decay dissipators arranged in a sunburst geometry. This configuration provides a unique perspective for understanding the interplay between dissipation and spatial distribution in open quantum systems.\\
    In the first part of the study, we focused on the behavior of the Liouvillian gap \( \Delta_\lambda \) as a function of the system size \( L \), under different scaling approaches. When the dissipator spacing \( b \) is fixed, the gap \( \Delta_\lambda \) remains finite and scales linearly with the dissipation strength \( w \). However, two distinct regimes emerge depending on the magnitude of \( w \): 
    1. For small \( w \), the gap is given by \( \Delta_\lambda = w/(2b) \).
    2. For large \( w \) and sufficiently large \( b \), it scales as \( \Delta_\lambda = wC_\mu/b^3 \).\\
    In the large-size limit, the latter regime dominates, and we derived scaling laws for \( L\Delta_\lambda \) as a function of \( wL \), which smoothly connects the two regimes. On the other hand, when the number of dissipators \( n \) is fixed, the gap \( \Delta_\lambda \) decreases as \( \sim L^{-3} \) for large \( L \). At small \( w \), a scaling law for \( L^2\Delta_\lambda \) in terms of \( wL \) reveals a non-uniform convergence of \( L^3\Delta_\lambda \) as \( w \to 0^+ \).\\
    In the second part, we developed a dynamic FSS theory at CQTs to describe the time evolution of the Kitaev model. For fixed \( b \), we extended the FSS framework to cases with \( b > 1 \). We introduced a scaling variable \( \gamma_b = wL^z/b \), supported by numerical results for two-point correlation functions and the entanglement entropy. Comparing real-time evolutions for rings with different \( b \) provided insights into dissipation mechanisms. Universal scaling functions were observed for entanglement entropy and \( P \)-correlations, but \( C \)-correlations showed non-universal behavior, warranting further investigation.\\
    For systems with a fixed number of dissipators \( n \), the FSS theory generalizes by replacing \( \gamma_b \) with \( W = wL^{z-1} \). We also explored the relationship between the Liouvillian gap \( \Delta_\lambda \) and the Kitaev gap \( \Delta \) in the FSS limit, analyzing the short- and long-time regimes. For fixed \( b \), these regimes transition smoothly, whereas at fixed \( n \), distinct power-law scalings emerge for \( \Delta \) and \( \Delta_\lambda \).
    \\$ $\\
    Finally, we address the case where the previous dissipative baths are replaced with thermal baths, studying their effects on the out-of-equilibrium dynamics of many-body systems within the quantum critical regime near a zero-temperature CQT.
    %Finally, we studied the effects of thermal baths on the out-of-equilibrium dynamics of many-body systems within their quantum critical regime near a zero-temperature quantum critical transition (CQT).
    Specifically, we analyzed the quantum evolution induced by quantum quenches (QQs) of Hamiltonian parameters under two distinct protocols involving a homogeneously coupled thermal bath.\\
    The first protocol, referred to as the unitary QQ protocol, uses the thermal bath to prepare the system at \( t = 0 \) in a finite-temperature Gibbs state. The subsequent dynamics, after the quench, are assumed to be unitary, with the thermal bath removed for \( t > 0 \). The second protocol, the dissipative QQ protocol, starts from the same initial Gibbs state but maintains the thermal bath throughout the evolution. This evolution is described by the Lindblad master equation, with the dissipative term modeling a thermal bath driving the system to a finite-temperature Gibbs state at large times. A key feature of this protocol is an additional time scale \( \tau = \gamma^{-1} \), where \( \gamma \) represents the decay rate of interactions between the system and the bath.\\
    Within the out-of-equilibrium FSS framework, we argued that when the temperature of the thermal bath is sufficiently small, its effects can be incorporated into extended zero-temperature scaling laws. For the unitary QQ protocol, the out-of-equilibrium FSS is achieved by rescaling the temperature as \( T \sim L^{-z} \), analogous to equilibrium FSS. For the dissipative QQ protocol, the dynamics is more complex because the decay rate \( \gamma \) plays an important role. To obtain a nontrivial FSS, both the temperature \( T \) and decay rate \( \gamma \) must be rescaled as \( T \sim L^{-z} \) and \( \gamma \sim L^{-z} \). Without rescaling \( \gamma \), the dynamics rapidly converge to equilibrium FSS at finite temperature after the critical time scale \( t_c \sim L^z \).\\
    These scaling arguments are supported by numerical results for the fermionic Kitaev model (equivalent to the quantum Ising chain) at its CQT separating disordered and ordered phases. Our model incorporates a thermal bath designed to guarantee asymptotic thermalization within the Lindblad framework. The derived scaling laws are general and are expected to apply to a wide range of many-body systems at CQTs interacting with homogeneous thermal baths in any spatial dimension.
    %When we keep $b$ fixed, the gap $\Delta_\lambda$ is always finite and depends linearly on the dissipation strength $w$. Nonetheless, two different regimes emerge for systems of finite size. In the small $w$ region, the gap is given by $\Delta_\lambda=w/(2b)$, whereas, at large $w$ and sufficiently large $b$, it behaves as $\Delta_\lambda=w C_\mu/b^3$. The last equation always controls the gap in the large-size limit and is our starting point to deduce the scaling of such a quantity when $b\propto L$. It is worth mentioning that we also put forward a scaling regime for $L\Delta_\lambda$ as a function of $wL$, which ties together the two different regimes outlined in a smooth manner. On the other hand, when we keep the number of dissipators $n$ fixed, the gap vanishes as $\sim L^{-3}$ at large $L$. Addressing the structure of the gap at small $w$, we find a scaling regime for $L^2\Delta_\lambda$ in terms of $wL$, which is closely related to the presence of a non-uniform convergence of $L^3\Delta_\lambda$ in the limit $w\to0^+$.\\$ $\\
    %We develop a dynamic FSS regime at CQTs to describe the time evolution of the Kitaev model under investigation. At fixed $b$, our results extend the FSS theory of Ref.~\cite{NRV-2019-competingdissipativeandcoherent} to the cases with $b>1$. As a working hypothesis, we suppose that the scaling variable associated with the relevant coupling $w$ is $\gamma_b=wL^{z}/b$. Our numerical results for the two-point correlation functions fully support this ansatz. When the number of dissipators $n$ is fixed, the FSS theory outlined at fixed $b$ generalizes straightforwardly after replacing $\gamma_b$ with $W=wL^{z-1}$. \\$ $\\
    %In the last section, we analyze the interplay between the Liouvillian gap $\Delta_\lambda$ and the gap related to the Kitaev ring $\Delta$ in the FSS limit. In particular, we take into account the short- and long-time regimes, focusing on how they join together in the FSS limit. When $b$ is fixed, we observe that the link between the two regimes is smooth, whereas, at fixed $n$, the two regions can be easily distinguished given the presence of different power-law scalings for the gaps $\Delta$ and $\Delta_\lambda$. 
}

