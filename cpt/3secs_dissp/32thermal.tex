\section{Thermal-bath effects in quantum quenches}

{\bf Breaf description of the model set: quantum quench protocol and observables}

The role of the temperature becomes less definite when
we consider out-of-equilibrium behaviors, because the
temperature of the system is an equilibrium concept.
However, one may consider the effects of thermal baths
in contact with the system during its out-of-equilibrium
dynamics. The main feature of a thermal bath is that
it eventually drives the system toward thermalization at
its temperature $T$ , in the large-time limit of the 
evolution of the system in contact with the thermal bath.


The thermalization process must somehow introduce a further
time scale $\tau$ in the problem, characterizing the approach
of the system to the thermal state when it is put in contact 
with the thermal bath. Such time scale is expected
to play an inportant role in the out-of-equilibrium dynamic
s of the system in contact with the thermal bath.
In this section we investigate these issues within the simplest
dynamic protocols giving rise to out-of-equilibrium
behaviors, i.e. those entailing instantaneous Quantum Quench (QQ) of the
Hamiltonian parameters starting from equilibrium thermal conditions.\\

To study the effects of a thermal bath in the out-of-equilibrium
behavior arising from a QQ within the critical regime,
we consider two protocols where the thermal
baths are involved in different ways:
\begin{itemize}
	\item
		Within the first protocol the thermal bath is used to
		prepare the system in a finite-temperature Gibbs state,
		described by the thermal density matrix (hereafter we set
		the Boltzmann constant $k_B = 1$)
X
%ρ t (w i , T ) =
%e −E n (w i )/T |Φ n , w i ihΦ n , w i |,

n
		where $\ket{\Psi_n, w_i}$ are the eigenstates of $H(w_i )$. Then the
		quantum evolution after the quench of the Hamiltonian
		parameters at $t = 0$ is unitary and driven by the Hamiltonian $H(w)$
		only, i.e., the thermal bath is removed during
		the quantum evolution for $t > 0$. Therefore, the evolu-
		tion of the density matrix is driven by the equation
%∂ t ρ(t) = −i[ Ĥ(w), ρ(t)],
%ρ(t = 0) = ρ t (w i , T ). (3)
\item In the second protocol the starting point is the
same, i.e. the Gibbs state (2), but the thermal bath
is not removed after quenching. Therefore, the out-of-equilibrium 
quantum evolution for $t > 0$ is not unitary
anymore, but it is also driven by the interaction with
the thermal bath. Under some conditions, discussed in
Refs. [5, 79–84], the nonunitary evolution arising from
the thermal baths can be described by a Lindbald master
equation governing the time evolution of the density
matrix of the system, which can be written as


%∂ t ρ = L[ρ] ≡ −i Ĥ(w), ρ + γ D T [ρ],
(4)
where ${\cal L}$ is a Liouvillian superoperator, and $D_T$ is a dissipative
driving whose strength is controlled by the homogeneous
coupling $\gamma$, playing the role of the decay rate
(inverse time scale) associated with the interactions between
the system and the bath. The operator $D_T$ is assumed
to be such that the Lindbald master equation (4)
drives the system toward an equilibrium Gibbs state at
temperature $T$ in the large-time limit.
\end{itemize}

\subsection{Out-of-equilibrium Scaling}


\subsection{Summary}
