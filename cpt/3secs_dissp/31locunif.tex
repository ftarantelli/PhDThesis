\section{From local to uniform dissipation: the sunburst model}

{\bf describe a bit the model and the interplay between uniform and local dissipation 
regime}

In this work, we focus on the case of particle-decay jump
operators, i.e., $\hat L_x = \hat c_x$ , where fermionic particles are
continuously removed from the site $x$. With this choice,
the Liouville operator ${\cal L}$ is quadratic in the fermionic
variables $\hat c_x$ and $\hat c_x^\dagger$ , and, in this sense, we say that the
open ring we study maintains its integrability. Most of
the results discussed in this work should preserve their
validity also for particle-pumping dissipation ( $\hat L_x = c_x^\dagger$ ),
since Eq. (3) is still quadratic in the fermionic creation
and annihilation operators.



\subsection{Liouvillian Gap}

This section is devoted to discussing the different scal-
ing behaviors observed for the Liouvillian gap ∆ λ . As
mentioned in the introduction, we will consider two dif-
ferent limits, depending on the number of dissipation
sources considered with increasing the lattice size. We
first review some useful definitions related to the Liou-
villian gap after rephrasing Eq. \eqref{eqlindblad} into a standard eigen-
value problem. To this purpose,


\subsection{Dynamic FSS Framework}







\subsection{Summary}
