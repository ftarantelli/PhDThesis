\section{Second-Order Transition}
\label{rtripcqt}

\subsection{Protocol}

KZ-like protocols have been largely employed to investigate the
critical dynamics of closed systems, subject to unitary time
evolutions.  To this purpose, one can assume that quasi-adiabatic
passages through the QT are obtained by slowly varying $h$ across $h_c
= 0$, following, e.g., the standard procedure:

\begin{itemize}
\item[$\bullet$] One starts from the ground state of the many-body system at
  $h_i < 0$, given by $|\Psi(t=0)\rangle \equiv |\Psi_0(h_i)\rangle$.
  
\item[$\bullet$] Then the out-equilibrium unitary dynamics, ruled by the
  Schr\"odinger equation
  \begin{equation}
    {{\rm d} \, |\Psi(t)\rangle \over {\rm d} t} =
    - i \, \hat H[h(t)] \, |\Psi(t)\rangle \,,
    \label{unitdyn}
  \end{equation}
  arises from a linear dependence of the time-dependent parameter
  $h(t)$, such as
  \begin{equation}
    h(t) = t/t_s \,,
    \label{wtkz}
  \end{equation}
  up to a final value $h_f>0$. Therefore the KZ protocol starts at
  time $t_i = t_s \, h_i<0$ and stops at $t_f= t_s \, h_f>0$.  The
  parameter $t_s$ denotes the time scale of the slow variations of the
  Hamiltonian parameter $h$.

\end{itemize}

Note that, in the case of one-dimensional quantum Ising model, the
slow variation of the longitudinal field $h$ across the CQT point
brings the system from a gapped condition at $h_i<0$ to another gapped
condition for $h_f>0$, i.e. move the system from disorder to disorder
through a CQT. Therefore, this is not the standard situation of the KZ
problem related to the defect production going from the disorder to
the order phases.

The resulting out-of-equilibrium evolution of the system can be
investigated by monitoring observables and correlations at fixed time.
In the case of Ising models in the presence of a time-dependent
longitudinal field $h(t)$, cf. Eq.~(\ref{iring}), one may consider the
evolution of the local and global average magnetization
\begin{equation}
  m_x(t) \equiv \big\langle \hat \sigma_x^{(1)}
  \big\rangle_t \,, \qquad M(t) \equiv {1\over L} \sum_x m_x(t)\,,
  \label{magnt}
\end{equation}
as well as the fixed-time correlation function of the order-parameter
operator and its space integral,
\begin{equation}
  G(t,x_1,x_2) \equiv \big\langle \hat \sigma_{x_1}^{(1)}
  \, \hat \sigma_{x_2}^{(1)} \big\rangle_t\,.
  \label{twopointt}
  \end{equation}
In the above formulas~\eqref{magnt} and~\eqref{twopointt}, $\langle
\,\hat O \,\rangle_t$ denotes the expectation value of the operator
$\hat O$ at time $t$. In the case the system is in a pure state
$|\Psi(t)\rangle$, this is given by $\langle \hat O \rangle_t \equiv
\langle \Psi(t)| \hat O |\Psi(t) \rangle$.  In the case the system is
in a mixed state $\rho(t)$, as when starting at finite temperature,
this is given by $\langle \hat O \rangle_t \equiv {\rm Tr} \, \big[
  \hat O \, \rho(t) \big]$.

To characterize the departure from adiabaticity along the slow dynamic
across the CQT, we also monitor the adiabaticity function
\begin{eqnarray}
  A(t) = - \ln  |\langle \, \Psi_0[h(t)] \, | \, \Psi(t) \, \rangle|\,,
%  \approx 
%  1 - |\langle \, \Psi_0[h(t)] \, | \, \Psi(t) \, \rangle|\,,
  \label{wtfunc}
\end{eqnarray}
where $|\,\Psi_0[h(t)]\,\rangle$ is the ground state of the
Hamiltonian $\hat H[h(t)]$, i.e. at instantaneous values of $h(t)$,
while $|\,\Psi(t)\,\rangle$ is the actual time-dependent state
evolving according to the Schr\"odinger equation (\ref{unitdyn}).  The
adiabaticity function measures the overlap of the time-dependent state
with the corresponding ground state of the Hamiltonian at the same
$h(t)$. Since the protocol starts from the ground state associated
with $h_i=h(t_i)$, we trivially have $A(t_i) = 0$.  If the quantum
evolution is adiabatic, then $A(t)=0$ at any time.  In the general
case arising from the above KZ protocol, $A(t)$ is expected to depart
from the initial value due to the impossibility of the system to
adiabatically follow the changes of the function $h(t)$ across its
critical value $h=0$.  Note however
that this is strictly true in the infinite-volume limit.  In system of
finite size $L$, there is always a sufficiently large time scale
$t_s$, so that the system can evolve adiabatically, essentially
because finite-size systems are always gapped, although the gap
$\Delta$ at the CQT gets suppressed as $\Delta \sim L^{-z}$. The
interplay between the size $L$ and the time scale $t_s$ gives rise to
nontrivial out-of-equilibrium scaling behaviors, which can be studied
within finite-size scaling (FSS) frameworks~\cite{RV-21,RV-20}.



\subsection{Dynamic Finite Size Scaling}





\subsection{Numerical Results}





\subsection{Summary}

