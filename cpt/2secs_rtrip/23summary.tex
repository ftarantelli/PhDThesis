

\section{Summary}

\rev{
We analyze the effects of a round-trip KZ protocol across 
a CQT and FOQT. We observe peculiar behaviors when 
we return to our initial state after the unitary 
time-evolution.\\$ $\\
In the CQT case, we address these issues within many-body model
undergoing quantum transitions, exploiting a
unified RG framework, where general dynamic scaling
laws are derived in the large-$t_s$ and large-$L$ limits. In particular,
we extend the RG framework already developed for standard KZ protocols.\\$ $\\
The observation of 
scaling behavior along the return way turns out to be more
problematic, due to the persistence of rapidly 
oscillating relative phases between the relevant quantum states.
They make the return way extremely sensitive to the 
parameters of the protocol, such as the extreme value $w_f$
and the size $L$ of the system. This is essentially related
to the quantum nature of the dynamics. Indeed there are
some notable similarities with the behavior of quantum
two-level models subject to round-trip protocols, 
analogous the well-known Landau-Zener-Stückelberg problem~\cite{tarantelli2022out}.\\$ $\\
In the ladder case, we introduce a 
}