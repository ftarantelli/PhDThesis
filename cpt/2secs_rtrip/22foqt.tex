\section{First-Order Transition}
\label{rtripfoqt}

\subsection{Driving Protocol}


Under this analogy, as well as the Kibble-Zurek mechanism captures the defects 
density generated across the criticality from an initial equilibrium homogeneous state,
out-of-equilibrium finite-size scaling relations at the FOQT quantify the transition to the
first excited level during the driving from an initial (non-critical) ground state.
However, understanding whether similar scaling relations occur when the system is
driven across a quantum phase transition from an out-of-equilibrium configuration is 
still not clear. So far, results are limited to the recent Ref. [53], and yet unexplored
for FOQTs. This is the scope of this manuscript. Below, we
investigate the emergence of finite-size scaling behaviors
during a round-trip driving across the first-order point. As
result, we find that out-of-equilibrium scaling behaviors
are still observed (even after several passages across the
FOQT), although the associated scaling functions develops
a dependence on the details of the driving protocol at the
inversion time.



\subsection{Finite Size Scaling at FOQT}







\subsection{Effettive Description}




\subsection{Floquet Driving}




\subsection{Summary}
