\chapter{Introduction}









\subsubsection{Outline}


for the chapter 2 we use the Quantum Ising Model as paradigmatic model, while in the third
one, we adopt the lattice Kitaev model. The choice is justified by the fact that for an 
unitary time-evolution is sufficient to know the wave function $\ket \psi$, instead in 
the case of a dissipation process, it is necessary to consider the density matrix $\rho$.

The Kitaev model is quadratic in its site operator, therefore the complexity of the 
diagonalization scales linearity with the system size. In the other hand, for the 
Quantum Ising model in the presence of longitudinal magnetization, the diagonalization
problem is exponentially complex. For these computational arguments, we could motivate
our toy model choices.
