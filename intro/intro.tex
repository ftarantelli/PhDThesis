\chapter{Introduction}

The progress achieved in the control of nano-scales many-body systems has recently renewed the interest in understanding the out-of-equilibrium dynamic in quantum spin models~\cite{PSSV-2011-noneqcoll, GAN-2014-quantumsimulation}. Out-of-equilibrium, these efforts provided, for instance, a characterization of the unusual spreading of correlations and entanglement \cite{kormos2017real,lerose2020quasilocalized,tortora2020relaxation,lagnese2022quenches,scopa2022entanglement,castro2020entanglement,vovrosh2021confinement,rigobello2021entanglement}, as well as of the thermalization \cite{birnkammer2022prethermalization,james2019nonthermal,robinson2019signatures,chanda2020confinement}, in condensed-matter analogs of confined systems.

A deeper comprehension of the time evolution of the critical correlations and entanglement spreading is indeed sought by both the theoretical and experimental communities~\cite{ADM-2015-EntanglementReview}.
Across phase transitions, the many-body system develops large-scale critical modes also
in out-of-equilibrium phenomena even when the time scale of the system parameters variation
is taken very large.
Out-of-equilibrium dynamic phenomena at
phase transitions, such as hysteresis and coarsening,
Kibble-Zurek (KZ) 
\cite{kibble1976topology,kibble1980some,zurek1985cosmological,zurek1996cosmological}
defect production, aging, etc., have
been addressed in a variety of contexts, both experi-
mentally and theoretically, at classical and quantum
phase transitions (see, e.g., Refs. 
\cite{binder1987theory, cui2020experimentally, bray2002theory, weiler2008spontaneous,
dziarmaga2010dynamics, PSSV-2011-noneqcoll, ulm2013observation} 
and references therein). Out-of-equilibrium scaling behaviors generally
emerge when slowly crossing a critical point, i.e. in
the large scale limit. They depend on the nature of the
classical or quantum transition, its universality class,
and the type of critical dynamics in classical systems, see e.g. Refs. 
\cite{kibble1980some, zurek1996cosmological, dziarmaga2010dynamics, PSSV-2011-noneqcoll}. 
Therefore, slow (quasi-adiabatic) passages through critical points allow us to
probe the universal features of the long-range modes
emerging at thermal and quantum critical phenomena \cite{tarantelli2022out}. 

In particular, the first part of this
work analyzes the application of a round-trip KZ protocol across a continuous quantum
phase transition. We take as paradigmatic model a Kitaev chain \cite{Kitaev_2001} in 
which the chemical potential changes linearity in time, following the function $t/t_s$ 
with the time $t$ and the time scale $t_s$.\\

Since any experimental device is unintentionally coupled to the environment, a particular emphasis is put on the dynamics of \textit{open quantum systems}~\cite{BP-openquantumsystembook}.

When the interactions of a quantum system with its surroundings are sufficiently weak, the real-time evolution of such apparatuses emerges from the interplay between the unitary and dissipative dynamics of the whole setup~\cite{RV-2021-coherentanddissipativedynamicsreview}. These hypotheses are usually satisfied within \textit{Lindblad} frameworks, which underpin the modelization of most atomic, molecular, and optical devices (AMO)~\cite{BDS-2015-KeldyshOptical}. In such cases, the system is described in terms of a density matrix $\rho$, and the time evolution is controlled by \textit{Linblad Master equations}
\begin{equation}
    \frac{d\rho}{dt} = \mathcal{L}[\rho]\,.
    \label{eq_def_intro_lindblad}
\end{equation}
The system generally thermalizes to a Non-Equilibrium Steady-State (NESS) solution after a transitory time frame. However, determining whether the NESS is unique is a more subtle issue~\cite{N-2019-uniquenesslindblad, SW-2010-openuniquesolution}. A quantity of particular interest is the \textit{Liouvillian gap}, hereafter denoted as $\Delta_\lambda$. This energy scale sets the typical relaxation time required to make the NESS stand out, entailing a complete loss of information on the initial quantum state. Quantum memory devices, for example, would benefit from long relaxation times, therefore small $\Delta_\lambda$~\cite{CCP-2011-quantummemories}.

Several works have addressed the nature of the Liouvillian gap in one-dimensional open quantum systems, considering different lattice geometries and dissipation sources also in integrable models~\cite{Z-2015-relaxtimes}. Distinguished behaviors emerge when the dissipators are either isolated or in a relatively large number compared to the system size $L$.
On the one hand, with bulk dissipation acting on the whole network, the system is gapped in several paradigmatic spin chains, such as XX, XXZ, and Ising models~\cite{YWHWD-2021-artificialnetweork, Z-2015-relaxtimes, KS-2019-nonhermitiankitaevladder}. On the other hand, when the number of dissipative sources is constant, the Liouvillian gap generally vanishes with a distinctive power-law behavior in the thermodynamic limit, typically as $\sim L^{-3}$~\cite{KS-2020-boundarydephasing, TV-2021-dissipativeboundaries, Z-2011-XXXchaingap}.
The physical mechanisms tying together these two regimes are still unclear and are the main focus of the second part of this work \cite{franchi2023Liouvillian}.

In particular, we consider a lattice model tailored to unveil the crossover regime between the  dissipation schemes presented. We investigate a $(1+1)$-dimensional Kitaev ring with local particle-decay dissipators arranged in a \textit{sunburst} geometry~\cite{FRV-staticsunburst, FRV-timesunburst, MS-2022-sunburstquench}.
Starting the protocol in the proximity of a Continuous Quantum Transition (CQT), we study the out-of-equilibrium dynamic using Renormalization Group (RG) arguments and Finite-Size Scaling (FSS) frameworks~\cite{C-1996-ScalingandRG, RV-2021-coherentanddissipativedynamicsreview}.





\subsubsection{Outline}
$ $

In the \textbf{Chapter} \ref{chp_out}, we introduce all the equations and the definitions
useful to present the original results in the last two chapters. Indeed, we give a brief
intro on the quantum phase transitions and, then, we explain two different mechanisms
which send the system out-of-equilibrium regime: the Kibble-Zurek mechanism and the
Lindblad mechanism.\\

In the \textbf{Chapter} \ref{chp_round}, we address the effect of a round-trip Kibble-Zurek
protocol near a continuous quantum phase transitions, using the Finite Size Scaling 
framework. We will find results which diverge from the classic hysteresis cycle scenario.\\

In the \textbf{Chapter} \ref{chp_diss}, we analyze the effects of a local and a uniform
dissipation process on a second-order transition. We will focus on the interplay between
these two type of dissipation mechanism, trying to extract common properties and different
behaviors.



%\begin{comment}
%We use the Quantum Ising Model as paradigmatic model, while in the third
%one, we adopt the lattice Kitaev model. The choice is justified by the fact that for an 
%unitary time-evolution is sufficient to know the wave function $\ket \psi$, instead in 
%the case of a dissipation process, it is necessary to consider the density matrix $\rho$.

%The Kitaev model is quadratic in its site operator, therefore the complexity of the 
%diagonalization scales linearity with the system size. In the other hand, for the 
%Quantum Ising model in the presence of longitudinal magnetization, the diagonalization
%problem is exponentially complex. For these computational arguments, we could motivate
%our toy model choices.
%\end{comment}
