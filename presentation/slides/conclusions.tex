\section{Conclusions}

\begin{frame}
	\frametitle{Conclusions}
	{\bf In the round-trip model:}
	\begin{itemize}
      	\item
      		Analogy of the scaling behaviors at classical and quantum transitions 
      		is only partially extended to round-trip KZ protocols. 
      		Substantial differences emerge:
      		\begin{enumerate} 
      			\item
      				classical systems develop scaling hysteresis-like scenarios, 
      			\item
      				in quantum systems, the persistence of oscillating relative 
      				phases make the return way extremely sensitive to the 
      				parameters of the protocol;
		\end{enumerate}
		
      	\medskip
      	\medskip

      	\item
      		Even in the simple two-level quantum model, we have a similar 
      		behavior.
      \end{itemize}
      
      \medskip
      \medskip
      \medskip
      
      {\bf In the dissipation scenario:}
      \begin{itemize}
      	\item
      		When we keep $b$ fixed, the gap $\Delta_\lambda$ is always finite
      		 and depends linearly on the dissipation strength $w \pc$ 
      	\medskip
      	\medskip
      	\item 			
      		 Two different regimes: 
      		 \begin{enumerate}
      		 	\item In the small $w$ region, the gap is given by 
      		 	$\Delta_\lambda = w/(2b) \pc$

			\item At large $w$ and sufficiently large $b$,
			$\Delta_\lambda = wC_\mu /b^3$ controls the gap
			in the large-size limit and the dynamic FSS.
		\end{enumerate}

      \end{itemize}

\end{frame}
