\chapter{Conclusions}

We have studied the out-of-equilibrium behavior of
many-body systems when their time-dependent Hamiltonian parameters 
slowly cross phase transition points. In particular,  we present an exploratory study of 
out-of-equilibrium behaviors arising from round-trip protocols across 
first and second order quantum phase transitions and classical transitions.

The
analogy of the scaling behaviors for one-way KZ protocols
at classical and quantum transitions is only partially
extended to round-trip KZ protocols. Substantial differences
emerge, in particular when the extreme value
of the outward driving parameter variation is kept fixed and
finite in the large-t s limit. On the one hand, classical
systems show a well-defined dynamic scaling limit, 
developing scaling hysteresis-like scenarios, essentially because
the purely relaxational stochastic dynamics leads 
eventually to thermalization at fixed model parameters. On the
other hand, in quantum systems the observation of 
scaling behavior along the return way turns out to be more
problematic, due to the persistence of rapidly oscillating
relative phases between the relevant quantum states.
They make the return way extremely sensitive to the 
parameters of the protocol, such as the extreme driving parameter value
and the size of the system. This is essentially related
to the quantum nature of the dynamics. Indeed there are
some notable similarities with the behavior of quantum
two-level models subject to round-trip protocols, 
analogous the well-known Landau-Zener-Stückelberg problem.


In the FOQT scenario, we formulate the FSS regime
as the limit $L \to \infty$, where the
time-dependent expectation values of local observables
are proportional to quasi-universal FSS functions of the
variables $\tau = t/ u$ and $\nu = u\delta^2$. Here, the meaning
of quasi-universality stands for the residual dependence of
the FSS functions on the details of the driving protocol
at the inversion time. Numerical results for the many-body
system confirm the validity of our scaling hypothesis.
We further probe the validity of the FSS
regime using time-dependent perturbation theory, relating it
to the emergence of an effective two-level description which
involves the lowest two states near the FOQT. With this effective
description, we reduce the driving protocol to a series of Landau-Zener
transitions and we determine an analytical expression of the
FSS functions. Lastly, we extend the setup to the case of
periodic driving across the quantum FOQT, and we comment on
the validity of the FSS after several crossings.



%\rev{ADD THE HOLE}

In the second part, we consider
systems where particles are constrained within a limited
spatial region by hard walls, and systems where they are
trapped by a space-dependent potential, such as an 
effective harmonic potential. These issues are particularly
relevant for cold-atom experiments, where atoms are 
confined within a limited spatial region by external 
potentials. Within this class of confined particle systems,
we investigate the dynamic features arising from 
localized particle-loss dissipative mechanisms, which may be
controllable, or inevitably present, in the experimental
setup. The out-of-equilibrium evolution arising from the protocol considered
shows various dynamic regimes, which can be effectively
distinguished by relating them to dynamic FSS
limits corresponding to different time scales, which can
be associated with the gap of the Liouvillian gap of the
Lindblad equation, and the gap of its Hamiltonian driving.

To understand the role of this Liouvillian gap,
we analyze the interplay between the Liouvillian gap $\Delta_\lambda$ and the gap related to the Kitaev ring $\Delta$ in the FSS limit. In particular, we take into account the short- and long-time regimes, focusing on how they join together in the FSS limit.
When we
keep $b$ fixed, the gap $\Delta_\lambda$ is always finite and depends
linearly on the dissipation strength $w$. Nonetheless, two different regimes emerge for systems of finite size. In the small $w$ region, the gap is given by $\Delta_\lambda=w/(2b)$, whereas, at large $w$ and sufficiently large $b$, it behaves as $\Delta_\lambda=w C_\mu/b^3$. The last equation always controls the gap in the large-size limit and is our starting point to deduce the scaling of such a quantity when $b\propto L$. It is worth mentioning that we also put forward a scaling regime for $L\Delta_\lambda$ as a function of $wL$, which ties together the two different regimes outlined in a smooth manner. On the other hand, when we keep the number of dissipators $n$ fixed, the gap vanishes as $\sim L^{-3}$ at large $L$. Addressing the structure of the gap at small $w$, we find a scaling regime for $L^2\Delta_\lambda$ in terms of $wL$, which is closely related to the presence of a non-uniform convergence of $L^3\Delta_\lambda$ in the limit $w\to0^+$.

We, also, develop a dynamic FSS regime at CQTs to describe the time evolution of the Kitaev model under investigation. At fixed $b$, our results extend the homogenous dissipative FSS theory to the cases with $b>1$. As a working hypothesis, we suppose that the scaling variable associated with the relevant coupling $w$ is $\gamma_b=wL^{z}/b$. Our numerical results for the two-point correlation functions and the entanglement entropy fully support this ansatz. In the second stage, we compare the real-time evolution of several rings corresponding to different $b$ to get some additional insights into the dissipation mechanisms of these systems.  As far as our numerical capabilities allow us to conclude, the entanglement entropy and the $P$-correlations admit a universal scaling function for all $b$, but the $C$-correlations do not. This issue requires further investigations to be better understood. When the number of dissipators $n$ is fixed, the FSS theory outlined at fixed $b$ generalizes straightforwardly after replacing $\gamma_b$ with $W=wL^{z-1}$. In the last section, we analyze the interplay between the Liouvillian gap $\Delta_\lambda$ and the gap related to the Kitaev ring $\Delta$ in the FSS limit. In particular, we take into account the short- and long-time regimes, focusing on how they join together in the FSS limit. When $b$ is fixed, we observe that the link between the two regimes is smooth, whereas, at fixed $n$, the two regions can be easily distinguished given the presence of different power-law scalings for the gaps $\Delta$ and $\Delta_\lambda$.  


Finally, we have reported a study of the effects of thermal
baths to the out-of-equilibrium dynamics of many-body
systems within their quantum critical regime close to a
zero-temperature CQT.
Within FSS frameworks, we argue that, when the thermal baths are
associated with a sufficiently small temperature, their effects can be
taken into account by appropriate extensions of the zero-temperature
out-of-equilibrium scaling laws describing soft QQs of isolated
systems within the critical regime.  For the unitary QQ protocol,
where the thermal bath only determines the initial Gibbs state and the
evolution is unitary, a nontrivial OFFS limit is simply obtained by
rescaling the temperature as $T\sim L^{-z}$, similarly to equilibrium
FSS.  Along the dissipative QQ protocol, where the thermal bath is not
removed after quenching, the dynamics is more complicated, and the
decay rate $\gamma$ plays a relevant role.  Indeed, in addition to the
rescaling of the temperature $T$ associated with thermal bath, one
also needs to rescale $\gamma$ as $\gamma\sim L^{-z}$ to obtain a
nontrivial FSS. Otherwise, when keeping $\gamma$ fixed, the dynamics
converges toward the equilibrium FSS at finite temperature, which
happens suddenly after quenching with respect to the time scale
$t_c\sim L^z$ of the critical regime.  Therefore the scaling behavior
when keeping $\gamma$ fixed becomes somehow trivial, reproducing the
equilibrium FSS for any rescaled time $\Theta = L^{-z} t >0$ in the
large-$L$ limit.



%\rev{ADD THE THERMAL}

As future outlooks, we mention that the results of the round-trip KZ protocol and of the Liouvillian gap analysis can be extended in several directions. First of all, our studies can be generalized by considering thermal baths in the Lindblad formalism~\cite{tarantelli2023thermal} with a round-trip protocol which drives the time evolution. Alternatively, it would be interesting to understand how the different large-size limits considered in this thesis affect the Liouville gap and the FSS regime of open quantum models in higher dimensions. Despite the numerous challenges given by such a quest, we must say that this setting certainly offers attractive questions and new paradigms to be addressed. To name a few, we mention that the NESS, in more than one spatial dimension, may undergo a continuous phase transition, similar to a finite-temperature quantum system at equilibrium. For this reason, the evolutions of open quantum models in the short- and long-time regimes can be associated with different RG fixed points, entailing a more intriguing scenario in the FSS limit.

%We conclude by mentioning that the results for the round-trip KZ protocol are also extended
%for first order transitions \cite{tarantelli2023out} in which we can approximate the 
%many-body system as a 2-level model within the scaling regime.

%One can also analyze the role of the temperature which becomes less definite when
%we consider out-of-equilibrium behaviors, because the temperature of the system is an
%equilibrium concept. However, one may consider the effects of thermal baths
%in contact with the system during its out-of-equilibrium dynamics. The main feature of
%a thermal bath is that it eventually drives the system toward thermalization at
%its temperature $T$ , in the large-time limit of the evolution of the system in contact
%with the thermal bath~\cite{tarantelli2023thermal}.