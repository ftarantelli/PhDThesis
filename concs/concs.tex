\chapter{Conclusions}

We have studied the out-of-equilibrium behavior of
many-body systems when their time-dependent Hamiltonian parameters 
slowly cross phase transition points. In particular,  we present an exploratory study of 
out-of-equilibrium behaviors arising from round-trip protocols across quantum
phase transitions.

As paradigmatic model we choose the Kitaev model which is mapped in the transverse quantum
Ising model through a JW transformation. Indeed, all the results are valid also in the
case of the longitudinal Ising model~\cite{tarantelli2022out} where the longitudinal 
magnetic field drives the round-trip KZ protocol. Note that, in the case of one-dimensional
quantum Ising model, the slow variation of the longitudinal field across the CQT point 
brings the system from a gapped condition to another gapped condition, i.e. move the system
from disorder to disorder through a CQT. Therefore, this is not the standard situation of
the KZ problem related to the defect production going from the disorder 
to the order phases.

The emerging dynamic scaling scenario put forward
for round-trip KZ protocols across critical points is expected to hold for generic 
classical and quantum transitions separating phases with short-range correlations, in
any spatial dimension. Further investigations are called
for round-trip protocols between disordered and ordered
phases, when the ordered phase has gapless excitations.
Round-trip KZ protocols in these systems may show further interesting features.

Indeed, we believe that also the chaotic-like return way calls for further investigation.
Even in the simple two-level quantum model some features of the behavior along the
return way turn out not to be smooth~\cite{tarantelli2022out, tarantelli2023out}.
Indeed, they develop ample oscillations with larger and larger frequencies
when increasing the interval of the round-trip variation of the parameters.\\


In the second part, we have considered a Kitaev ring coupled with the environment via $n$ particle-decay dissipators arranged in a sunburst geometry.

We analyze the interplay between the Liouvillian gap $\Delta_\lambda$ and the gap related to the Kitaev ring $\Delta$ in the FSS limit. In particular, we take into account the short- and long-time regimes, focusing on how they join together in the FSS limit. When $b$ is fixed, we observe that the link between the two regimes is smooth, whereas, at fixed $n$, the two regions can be easily distinguished given the presence of different power-law scalings for the gaps $\Delta$ and $\Delta_\lambda$.\\

As future outlooks, we mention that the results of the round-trip KZ protocol and of the Liouvillian gap analysis can be extended in several directions. First of all, our studies can be generalized by considering thermal baths in the Lindblad formalism~\cite{tarantelli2023thermal}. Alternatively, it would be interesting to understand how the different large-size limits considered in this paper affect the Liouville gap and the FSS regime of open quantum models in higher dimensions. Despite the numerous challenges given by such a quest, we must say that this setting certainly offers attractive questions and new paradigms to be addressed. To name a few, we mention that the NESS, in more than one spatial dimension, may undergo a continuous phase transition, similar to a finite-temperature quantum system at equilibrium. For this reason, the evolutions of open quantum models in the short- and long-time regimes can be associated with different RG fixed points, entailing a more intriguing scenario in the FSS limit.

We conclude by mentioning that the results for the round-trip KZ protocol are also extended
for first order transitions \cite{tarantelli2023out} in which we can approximate the 
many-body system as a 2-level model within the scaling regime.

%One can also analyze the role of the temperature which becomes less definite when
%we consider out-of-equilibrium behaviors, because the temperature of the system is an
%equilibrium concept. However, one may consider the effects of thermal baths
%in contact with the system during its out-of-equilibrium dynamics. The main feature of
%a thermal bath is that it eventually drives the system toward thermalization at
%its temperature $T$ , in the large-time limit of the evolution of the system in contact
%with the thermal bath~\cite{tarantelli2023thermal}.


